Wir betreiben für euch eine Vielzahl von Mailinglisten, die euch im Studium helfen sollen. Außerdem nutzen wir diese Mailinglisten, um euch über wichtige Veranstaltungen und Themen auf dem Laufenden zu halten. Leider melden sich nicht immer alle Erstsemester auf unseren Mailinglisten an.\\
Vor allem unsere Mailingliste info-studium sollten alle abonnieren! Über diese schicken wir alles was alle Informatikstudierende angeht und Einladungen und Terminankündigungen für unsere Veranstaltungen wie unser Sommerfest oder unsere Clubhausfeste.\\
\\
Hier findet ihr nochmal eine Übersicht über die wichtigsten Mailinglisten. Weitere Mailinglisten könnt ihr auf \url{http://www.fsi.uni-tuebingen.de/studium/mailinglisten} finden.

\newcommand{\mladressen}[1]{
    Anmelden: Leere Mail an {\footnotesize \email{#1-subscribe@fsi.uni-tuebingen.de}} \\
    Abmelden: Leere Mail an {\footnotesize \email{#1-unsubscribe@fsi.uni-tuebingen.de}} \\
    Hilfetext: Mail mit Betreff \emph{help} an {\footnotesize \email{#1-request@fsi.uni-tuebingen.de}}}

\begin{description}

  \item[info-studium\At fsi.uni-tuebingen.de (Studium)] ~\\
    Hier geht es um alle Themen, die irgendwie mit dem Informatik-, Bioinformatik- oder Medieninformatik-Studium
    oder dem informatischen Teil der Kognitionswissenschaft zu tun haben. 

    \mladressen{info-studium}


  \item[info-jobs\At fsi.uni-tuebingen.de (Stellenangebote)] ~\\
    Wer auf der Suche nach einem Job ist, sollte sich auf dieser Verteilerliste
    für Stellenangebote anmelden. Achtung, auf dieser Liste sollten nur die
    Angebote selbst und keine Diskussionen darüber landen.

    \mladressen{info-jobs}

  \item[kogwiss\At fsi.uni-tuebingen.de (Kognitionswissenschaft)] ~\\
    Auf dieser Liste geht es um Themen, die lediglich Kognitionswissenschaftler
    betreffen.

    \mladressen{kogwiss}
  
  \item[versuche\At fsi.uni-tuebingen.de (Teilnahme an Versuchen)] ~\\
    Wer an wissenschaftlichen Versuchen und Studien teilnehmen will bzw. muss (z.B. Kognitionswissenschaftler,
    die Versuchspersonenstunden ableisten müssen), findet auf dieser 
    im WS15/16 neu eingerichteten Mailingliste Einladungen zur Teilnahme an Versuchen.
  
  \mladressen{versuche}
\end{description}
\vfill

%Für die Experten: Mails von diesen Listen filtern:

%Die beste Möglichkeit ist, den 'List-Id:'-Header zu verwenden.

