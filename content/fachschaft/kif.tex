Auch dieses Semester waren wieder Fachschaftsmitglieder auf der Konferenz der Informatik Fachschaften (KIF), der unserer Bundesfachschaftentagung. Auf der KIF kommen Fachschaften aus allen deutschsprachigen Ländern zusammen, um sich auszutauschen und Leute in verschiedene Gremien zu wählen.\\
In diesem Semester fand sie in vom 11.11.2015 bis zum 15.11.2015 in Bonn statt. Wir waren dort dieses mal mit drei Fachschaftlern vertreten.\\
Mittwoch Abend startete die KIF zunächst mit einem Anfangsplenum. Hier berichteten alle vertretenen Fachschaften und die von der KIF im vorhergegangenen Semester entsandten Vertreter kurz vor was seit der letzten KIF passiert ist. Anschließend wurden die Arbeitskreise, in denen die darauffolgenden Tagen gearbeitet und sich ausgetauscht wurde, vorgestellt.\\
Wir konnten in verschiedenen Arbeitskreisen wieder einige Ideen sammeln. Unter anderem enstand dabei auch die Idee für diesen Newsletter.\\
Samstagabend begann schließlich das Abschlussplenum. Hier wurden die Ergebnisse der Arbeitskreise vorgestellt, neue Vetreter in verschiedene Gremien entsandt oder Alte bestätigt. Abschließend wurde noch eine Resolution versabschiedet, in der wir uns kritisch gegenüber der Novelierung des sogenannten Wissenschaftszeitvertragsgesetz äußern.\\
Mit diesem Gesetzesentwurf soll die Situation der zahlreichen Wissenschaftler verbessert werden, die immer wieder nur befristet angestellt werden. Auch wenn wir diese Intention begrüßen, gefährdet die aktuelle Fassung das Prinzip der studentischen Hilfskräfte. Mit der beschlossenen Resolution fordern wir den Gesetzgeber auf an dieser Stelle noch einmal nachzubessern.


