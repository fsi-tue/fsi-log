Dieses Jahr durften wir wieder über 300 Erstsemester begrüßen. Durch die Hilfe zahlreicher Fachschaftsmitgliedern und weiteren Helfern konnten wir in diesem Semester unser Programm weiter ausbauen. Zusätzlich zu den seid Jahren stattfindenden Veranstaltungen konnten wir noch einen Spieleabend, eine Party und eine Wanderung organisieren.
%
%\textbf{Grillen}

Am Nachmittag der Vorkursklausur organisierte Felicia zum Ausgleich ein gemeinsames Grillen für die Vorkursler auf dem Sand. Ungefähr 100 Erstsemester wurden dabei von uns begrillt, sodass wir uns zu den gegebenen Sitzgelegenheiten weitere Bierbänke und -tische vom Lehrstuhl von Professor Zell ausleihen mussten.
%
%\textbf{Frühstück und Sandführung}

Traditionell fand am letzten Freitag vor dem Beginn der Vorlesungen wieder unser Anfängerfrühstück mit anschließender Führung über die Morgenstelle und verschiedene Lehrstühle auf dem Sand statt. 
Lea Buchweitz organisierte ...
%
%\textbf{Kneipentour}
%
%\textbf{Stadtführung}
%
%\textbf{Spieleabend}

Erstmalig veranstalteten wir in diesem Semester einen Spieleabend für die neuen Erstsemester. Dank der vielen Menschen, die eigene Spiele mitbringen, konnte aus einer Vielzahl von Spielen gewählt werden.
Durch den großen Andrang und viel positives Feedback haben wir beschlossen weitere Spieleabende auch außerhalb der Erstsemesterphase für alle veranstalten und Spiele hierfür anzuschaffen.\\
Der erste dieser Spieleabende fand am 25.11.2015 statt.
%
%\textbf{Party}

Neben den neuen Veranstaltungen organisierten David-Elias Künstle und Marc Weitz auch zum ersten mal seit 2009 wieder einen Klassiker der Erstsemesterveranstaltungen, die Anfiparty. Hierbei waren alle Erstsemester herzlich eingeladen, auf ein Bier oder einen Glühwein in gemütlicher Runde auf den Sand zu kommen. Anschließend ging es dann gesammelt zur Semester Opening Party des Kuckucks in die Mensa Morgenstelle.
%
%\textbf{Wanderung}

Eine weitere neue Veranstaltung in diesem Semester war einen gemeinsame Wanderung. Maximus Mutschler und Marc Weitz leiteten dabei den Ausflug nach Bebenhausen, wo anschließend das mittelalterliche Kloster besichtigt wurde. 

\textbf{Wochenende}

