\textbf{Dieses Jahr durften wir wieder über 300 Erstsemester begrüßen. Durch die Hilfe zahlreicher Fachschaftsmitgliedern und weiteren Helfern konnten wir in diesem Semester unser Programm weiter ausbauen. Zusätzlich zu den seid Jahren stattfindenden Veranstaltungen konnten wir noch einen Spieleabend, eine Party und eine Wanderung organisieren.}
%
%\textbf{Grillen}

Unsere Veranstaltungen starteten dieses Jahr wieder am Nachmittag der Vorkursklausur. Zum Ausgleich organisierten wir ein gemeinsames Grillen für die Vorkursler auf dem Sand. Ungefähr 100 Erstsemester wurden dabei von uns begrillt, sodass die vorhandenen Sitzgelegenheiten fast nicht ausreichten. Zum Glück konnten wir uns weitere Bierbänke und -tische vom Lehrstuhl von Professor Zell ausleihen, wofür wir uns an dieser Stelle nochmals herzlichst bedanken möchten.

%\textbf{Frühstück und Sandführung}
Weiter ging es am Freitag vor dem eigentlichen Semesterstart mit einem gemeinsamen Frühstück. Ziel ist ein gemütliches Zusammensitzen bei dem die Studierenden höherer Semester ihre Erfahrungen weitergeben und Erstsemester alle möglichen Fragen stellen können. An den langen Frühstückstafeln haben die Erstsemester außerdem die Möglichkeit sich gegenseitig schon ein wenig kennenzulernen.\\
In Zusammenarbeit mit der Mensa Morgenstelle konnten wir wieder für reichlich Kaffee, Tee, Saft, Brötchen und Belag sorgen. Traditionell werden die Erstsemester hier auch gleich von der Fachschaft und dem Professor der Informatik I begrüßt. Anschließend haben wir wieder Führungen über den Campus Morgenstelle angeboten, damit die Erstsemester schon einige Tage vor Beginn des Studienalltags die Umgebung kennenlernen können.\\
Am Nachmittag wurden die einzelnen Studiengänge nochmals gesondert von einigen Professoren auf dem Sand begrüßt. Anschließend bestand für die Erstsemester die Möglichkeit sich einige Arbeitsbereiche des Instituts genauer anzuschauen. Dank der Mithilfe zahlreicher Professoren konnten an mehreren Stationen die Arbeit der Lehrstühle begutachtet werden.
%
%\textbf{Kneipentour}
%
%\textbf{Stadtführung}
%
%\textbf{Spieleabend}

Zum ersten Mal veranstalteten wir in diesem Semester einen Spieleabend für die neuen Erstsemester. Dank der vielen Menschen, die eigene Spiele mitbringen, konnte aus einer Vielzahl von Spielen gewählt werden. In Kleingruppen wurde dann bis zum letzten Bus gespielt, gelacht und sich kennengelernt. Wir haben uns sehr über den großen Andrang und das positives Feedback gefreut und haben deshalb beschlossen, weitere Spieleabende auch außerhalb der Erstsemesterphase für alle Studierenden der Informatik anzubieten. Damit nicht jedes mal jeder Spiele mitbringen muss, haben wir eine kleine Auswahl angeschafft, die auch außerhalb von Spieleabenden von allen Informatikstudierenden benutzt werden darf
%\textbf{Party}

Neben den neuen Veranstaltungen organisierten wir auch zum ersten mal seit 2009 wieder einen Klassiker der Erstsemesterveranstaltungen, die Anfiparty. Hierbei waren alle Erstsemester herzlich eingeladen, auf ein Bier oder einen Glühwein in gemütlicher Runde auf den Sand zu kommen. Anschließend ging es dann gesammelt zur Semester Opening Party des Kuckucks in die Mensa Morgenstelle.
%wieder
%\textbf{Wanderung}

Eine weitere neue Veranstaltung in diesem Semester war einen gemeinsame Wanderung nach Bebenhausen, wo wir gemeinsam das mittelalterliche Kloster besichtigt haben.

%\textbf{Wochenende}
Den Abschluss der Erstsemesterveranstaltungen war das Anfiwochenende, welches wir mit den neuen Erstsemetern in einer Hütte im Umkreis von Tübingen verbringen.
Dieses Jahr ging die Reise nach Schachen, ein kleines Pfadfinderzentrum auf der schwäbischen Alp. Neben Aktionen wie Lagerfeuer, Nachtwanderung und Gruppenspielen im Freien war abends jede Menge Zeit sich beim gemütlichen Beisammensein kennenzulernen. 

Habt ihr coole Ideen, was wir noch an Veranstaltungen anbieten könnten? Oder habt ihr sogar Lust zu helfen? Dann kommt doch mal zu einer unserer Sitzungen
%\footnote{\url{http://www.fsi.uni-tuebingen.de/fachschaft/sitzung}} 
oder schreibt uns eine Mail
%\footnote{\email{fsi@fsi.uni-tuebingen.de}}
.

\textbf{Helft uns dabei, die nächsten Erstsemesterveranstaltungen noch besser und noch größer zu machen!}

\vfill
\begin{center}

	% Einrahmung der Minipage

	\fbox{%

		% erste Minipage mit einer Breite von 80% der Textbreite

		\begin{minipage}[t]{0.9\textwidth}

			\begin{center}

				\Huge Helfer gesucht! \normalsize

			\end{center}

			Wir wollen im nächsten Semester zum ersten Mal einen Workshopnachmittag für die neuen Erstsemester veranstalten. Wenn ihr Lust habt einen Workshop anzubieten, schickt uns einfach eine Mail an \email{fsi@fsi.uni-tuebingen.de}.\\

			Die Workshops sollen sich mit Themen befassen, die im Studium nicht direkt behandelt werden, aber vieles vereinfachen, wie z.B.

			\begin{itemize}

				\item Git

				\item \LaTeX

				\item Linux

			\end{itemize}
			\begin{center}
				\textbf{Wir freuen uns auf eure Ideen!}
			\end{center}
			\vspace{0.5cm}
			

		%	\begingroup

		%	\parfillskip=0pt

			% zwei weitere Minipages

		%	\Huge Helfer gesucht!

		%	\par\endgroup

		\end{minipage}

	}

\end{center}
