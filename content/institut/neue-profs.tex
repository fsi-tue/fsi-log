%An dieser Stelle möchten wir euch immer wieder die neuen Professoren hier am Institut vorstellen.
\large \textbf{Prof. Dr. Ulrike von Luxburg}\\
\normalsize Lehrstuhl für Theorie des Maschinellen Lernens\\
\normalsize

\begin{minipage}[h]{0.65\textwidth}
	\textbf{Kaffee oder Tee?}\\
	Tee.\\
	\textbf{Du oder Sie?}\\
	Studies: Sie, Mitarbeiter: Du.
	\\
	\textbf{Bleistift oder Kugelschreiber?}\\
	Bleistift.
	\\
	\textbf{Compilersprachen oder Scriptsprachen?}\\
	Beides, zur Zeit mehr script.
	\\
	\textbf{GUI oder Terminal?}\\
	Terminal.
	\\
	\textbf{Beamer, Tafel oder Whiteboard?}\\
	Beamer und Tafel.
	\\
	\textbf{Windows, Linux oder OS X?}\\
	OS X.
	\\
\end{minipage}
\begin{minipage}[h]{0.35\textwidth}
	\vspace{0.5cm}
	\includegraphics[width=\textwidth]{content/pictures/luxburg.jpg}
	Quelle: Universität Tübingen/Friedhelm Albrecht
\end{minipage}

\textbf{Was ist ihre Horrorvorstellung von einer Vorlesung?}\\
Horrorvorstellung?!? Habe ich keine. Aber es ist nervig, wenn die Studies nicht mitdenken und keine Lust haben, meine Fragen zu beantworten und selber Fragen zu stellen. Dann kann man genauso gut gegen eine Wand reden, und das macht keinen Spaß. 

\textbf{Was ist ihre beliebteste Klausurfrage?}\\
Kommt natürlich auf die Vorlesung an. Ein Klassiker im Bereich Maschinelles Lernen: Sie sehen eine Punktmenge und wie sie klassifiziert wird. Welcher Algorithmus hat dieses Ergebnis erzeugt, und mit welchen Parametern (groß, klein)? Warum? 

\textbf{Wie sieht die perfekte Antwort auf diese Frage aus?}\\
Hängt vom Bild ab! 

\textbf{Welche Vorlesung wollen Sie auf gar keinen Fall lesen? Und warum?}\\
Vorlesungen sind auch für Dozenten ein guter Weg, mal wieder was neues zu lernen. Daher würde ich nichts prinzipiell ausschließen. Aber bei vielen Themen gibt es sicher Leute am Fachbereich, die dafür eher prädestiniert sind als ich ... 